\documentclass{beamer}
\usepackage{ragged2e}
\usepackage{hyperref}
\usetheme{Madrid}
\useinnertheme{circles}
\title{Srinivas Ramanujan}
\author{Deepika Gupta }
\institute{Ramanujan College}
\date{October 2021}
\begin{document}
\maketitle
\subtitle{Introductions}
\begin{frame}{Contents}
 \begin{itemize}
     \item About
     \item Works
     \item Achievement
 \end{itemize}
\end{frame}
\begin{frame}{About}
 \textbf{Srinivasa Ramanujan,} (born December 22, 1887, Erode, India—died April 26, 1920, Kumbakonam), Indian mathematician whose contributions to the theory of numbers include pioneering discoveries of the properties of the partition function.
\end{frame}
\begin{frame}{Works}
\begin{itemize}
    \item In 1911 Ramanujan published the first of his papers in the \textit {Journal of the Indian Mathematical Society}. His genius slowly gained recognition, and in 1913 he began a correspondence with the British mathematician Godfrey H. Hardy that led to a special scholarship from the University of Madras and a grant from Trinity College, Cambridge.
    \item 
    \begin{equation}
      \frac{1}{\pi}=\frac{\sqrt{2}}{9801} \sum_{n=1}^{\infty}{4 \fact{n}}
    \end{equation}
 \end{itemize}
\end{frame}
\end{document}
